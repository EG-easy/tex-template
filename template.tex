\documentclass{jsarticle}
\usepackage{bm}
\usepackage{fancyhdr}
\usepackage{amsmath}
\usepackage{amssymb}
\usepackage{braket}
\usepackage{booktabs}
\usepackage{array}
\usepackage{graphicx}
\usepackage{here}
\usepackage{multicol}
\usepackage{okumacro}
\usepackage[dvipdfmx]{hyperref}
\usepackage{pxjahyper}

% ソースコードを貼るとき
% pip install pygments
% \usepackage{minted}

%\usepackage{txfonts}
%\pagestyle{empty}
\pagestyle{fancy}
\rhead{templete}


\begin{document}

\title{TEMPLETE}
\author{Mr. temple}
\date{\today}
\maketitle

%\newpage
%\tableofcontents
%\newpage

\section*{}
This is a templete file.

 \section*{1}
Hello, world.
\subsection*{1}
Hi, Temple!


\end{document}

%表の挿入
\begin{table}[H]
\begin{center}
\caption{表のタイトル}
\begin{tabular}{ccc} \hline
 &  &  \\ \hline
 &  &  \\
 &  &  \\
 &  &  \\
 &  &  \\ \hline
\end{tabular}
\end{center}
\end{table}


%図の挿入
\begin{figure}[H]
\begin{center}
\includegraphics[width = 12.0cm, bb= 0 0 360 252]{ファイル名.eps}
\end{center}
\caption{グラフのタイトル}
\label{fig:}
\end{figure}


%図を2つ並列
\begin{figure}[H]
\begin{minipage}{0.50\hsize}
\begin{center}
\includegraphics[width = 4.0cm,bb=0 0 600 600]{.ps}
\end{center}
\caption{タイトル}
\label{fig:}
\end{minipage}
\begin{minipage}{0.50\hsize}
\begin{center}
\includegraphics[width = 4.0cm,bb=0 0 600 600]{.ps}
\end{center}
\caption{タイトル}
\label{fig:}
\end{minipage}
\end{figure}



%図を3つ並列
\begin{figure}[H]
\begin{minipage}{0.33\hsize}
\begin{center}
\includegraphics[width = 4.0cm,bb=0 0 600 600]{.ps}
\end{center}
\caption{タイトル}
\label{fig:}
\end{minipage}
\begin{minipage}{0.33\hsize}
\begin{center}
\includegraphics[width = 4.0cm,bb=0 0 600 600]{.ps}
\end{center}
\caption{タイトル}
\label{fig:}
\end{minipage}
\begin{minipage}{0.33\hsize}
\begin{center}
\includegraphics[width = 4.0cm,bb=0 0 600 600]{.ps}
\end{center}
\caption{タイトル}
\label{fig:}
\end{minipage}
\end{figure}


%文章と図の2段組
\begin{multicols}{2}
文章

\begin{figure}[H]
\begin{center}
\includegraphics[width = 12.0cm, bb= 0 0 360 252]{.eps}%図
\end{center}
\caption{タイトル}
\label{fig:1.1.1}
\end{figure}
\end{multicols}

% ソースコードを入れる
\begin{minted}[linenos,
               numbersep=5pt,
               frame=lines,
               framesep=2mm]{c}
#include <stdio.h>

int main(void) {
    printf("Hello, world!\n");
    return 0;
}
\end{minted}
